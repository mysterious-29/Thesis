%%%%%%%%%%%%%%%%%%%%%%% CHAPTER - 1 %%%%%%%%%%%%%%%%%%%%\\
\chapter{Introduction}
\label{C1} %%%%%%%%%%%%%%%%%%%%%%%%%%%%
\graphicspath{{Figures/PDF/}{Figures/PNG/}}
\noindent\rule{\linewidth}{2pt}

%%%%%%%%%%%%%%%%%%%%%%%%%%%%%%%%%%%%%%%%%%%%%%%%%%%%%%%%%%%%%%%%%%%%%%%%%%%%%%%%%%%%%%%%%%%%%%%%%%%%%%%%%%%%%%%%%%%%%%%%%%%%%%%%%%%%%%
% \section{General Introduction} \label{S1.1} 
\noindent Wireless Sensor Network has a number of sensor nodes. These sensor nodes are having the property of auto-configuration and self-organization. Network architecture follows decentralization and is distributed in nature. WSN has a number of application in the areas of environmental monitoring like air, water and soil, structural monitoring, human activity, and behavioral monitoring, military surveillance, asset tracking and many more \cite{akyildiz2002wireless}. The advantage of WSN nodes is their small size and these can be placed at some place where surveillance by a wired network or human is not possible. These small nodes are deployed in thousands in number to monitor temperature, pressure etc. These nodes are always prone to physical damage because these are deployed in the open and unprotected environment. Self-organizing and auto-configuration in nature, limited battery, computation power, and bandwidth, distributed and decentralization, multihop communication in wireless medium, are some characteristics which may lead to exposing this network to many security attacks in all OSI model layers. 
% In order to provide protection to wireless sensor networks, many solutions such as cryptographic and secure routing, key exchange and authentication are proposed. These methods are used to provide security from outside attack up to some level but these cannot eliminate all security attacks \cite{ping2008distributed}. 
To detect an inside attack Intrusion Detection System is introduced which can deal with wide range attacks in WSN.
\par
Security attack in wireless sensor network is either active or passive. In passive attack attacker generally, hide and collects data by tapping communication link but does not modify it. Traffic analysis, malfunctioning of a node, eavesdropping comes under passive attack. The active attack affects the operation of the network. An active attack can lead to termination or degradation in networking services. Denial-of-service (DoS, Ddos), Network jamming, wormhole, black hole, sinkhole attacks are grouped in active attacks \cite{padmavathi2009survey}. Solutions to these attacks (active or passive) in any network involves Prevention, Detection, and Mitigation \cite{fuchsberger2005intrusion}. In prevention step, the technique provides defense mechanism against attack i.e. it prevents the attack. Detection step comes into action when an attacker has found a way to pass prevention technique. This involves to detect or identify the node which has been attacked and compromised (being aware of the presence of attack). Mitigation aims to remove the attack detected in the detection step by taking action against the compromised node.
\par
IDS functionality is defined by the detection method which are mainly of two types- Anomaly-based, Misuse-based. Anomaly-based technique tries to finds the deviation from normal behavior. To flag operation as an anomaly, the regular observation of system must be there to accommodate system changes. Misuse-based technique tries to detect previously known attack with high detection rate by comparing the new attack signature with known signatures.
\section{Motivation}
Application domain of WSN is very wide in the fields of Medical to military and data plays very important role in every field. Intrusion is any activity in a network which is not authorized which affects network's services, resources or data either passively or actively. Such activity if not prevented in the first line of defense in WSN security then IDS comes into play as the second line of defense. Intrusion detection is performed by the network members nodes by detecting any suspicious behavior. After detection, it cannot take action on it but it raises an alarm to the controller. IDS provides information to the controller about intruder information (identification of node or region, location, time and date of intrusion, an activity of intrusion, type of intrusion etc). This information is used in the third line of defense i.e. in mitigating the attack.

\section{Problem Statement and Objectives}
In order to provide protection to WSNs many solutions such as cryptographic and secure routing, key exchange and authentication are proposed. These methods are used to provide security from outside attack up to some level but these cannot eliminate all security attacks. To detect an inside attack IDS is introduced which can deal with wide range of attacks in WSN. Main motive of this research is to design an IDS which can deal with intrusion attacks efficiently. Main focus in this research is been on high detection rate \& low false positive rate.
\section{Organization of Dissertation}
The dissertation is structured in chapters. Organization of chapters are as follows:
\\
\noindent Chapter \ref{C1} presents a brief introduction about WSN and IDS, motivation for this research work, problem statement and objectives of this research.
\par Chapter \ref{C2} gives a detailed overview of WSN followed by attacks in WSN. Various security attacks in wireless sensor network like DoS attack, replay attack, traffic analysis, blackhole, wormhole, sinkhole attacks etc has been discussed. Also, it is shown that which type of attack may occur on which layer of the OSI model. Different classes and types of attack are discussed in great detail. Then limitations and challenges faced by WSN are discussed. After that, brief introduction to applications of WSN is provided. 
\par Chapter \ref{C3} provides description of IDS. This chapter starts with introduction of IDS and it's components. After that, classification  of IDS based on detection method, source of data etc. is discussed in great detail.
\par Chapter \ref{C4} discesses a detailed literature survey of the IDS.
Various approaches of detection are discussed which are based on different fields like Neural network, support vector machine, cluster-based approaches etc. A tabular comparison of some detection techniques is also presented in this chapter which includes limitations and features of particular approach.
\par Chapter \ref{C5} is focused on the proposed system for attack detection. It starts with the system model of proposed approach followed by dataset description used for training purpose. After that, proposed approach of neural network model is discussed. A systematic flow chart of algorithm is also discussed in this chapter. Steps of detection model are described in procedure section. This chapter ends with summary of proposed system.
\par Chapter \ref{C6} presents results of research work. It starts with discussing the simulation environment briefly followed by 3 comparison graphs of WSN energy model. After that, this chapter presents the 2 accuracy graphs of neural network model.
\par Chapter \ref{C7} concludes the dissertation. Conclusion explains the main findings and limitation of this research work with scope of future work.