%%%%%%%%%%%%%%%%%%%%%%% CHAPTER - 7 %%%%%%%%%%%%%%%%%%%%\\
\chapter{Conclusions and future directions}
\label{C7} %%%%%%%%%%%%%%%%%%%%%%%%%%%%
%\graphicspath{{Figures/Chapter-6figs/PDF/}{Figures/Chapter-6figs/}}
\noindent\rule{\linewidth}{2pt}
%%%%%%%%%%%%%%%%%%%%%%%%%%%%%%%%%%%%%%%%%%%%%%%%%%%%%%%%%%%%%%%%%%%%%%%%%%%%%%%%%%
% \textit{The research work presented .................................}
Applications of wireless sensor networks are so wide that WSN is reaching in all areas from military to medical. In all applications, data plays an important role and hence it must be kept secure. Various possible attacks try to modify data or analyze it. An outsider attack can be detected or prevented using Cryptographic and other security mechanisms. When the attacker resides inside the network, in this situation the node which is compromised and under attack can lead to various security problems. Using this node any data/information passing through this node can be manipulated. So, it is important to detect these inside attacks and prevent any data loss. Intrusion detection systems are designed for this purpose.
\par Some of the IDS system are able to detect known attack while some were able to detect unknown attacks too. Hybrid systems have also been designed in order to detect all kinds of security attacks in the wireless sensor network. Accuracy and high detection rate of some of these detection mechanisms were about 100\% and hence low false positive rate. Some IDS can detect all the attacks, while on the other hand some of those were designed to detect specific attacks efficiently. Features and applications of different detection methodologies have been compared. Conclusively, this survey paper concludes various types of IDS. In the process of analyzing different detection techniques, it is observed that anomaly-based detection methods normally use statistical algorithms whereas signature-based detection methods generally uses knowledge-based algorithms. Also, it can be inferred that hybrid systems are more efficient in terms of detection rate with low false positive rate as compared to the anomaly and signature-based techniques. The work presented in this research can be summarized as follows:
\section{Conclusions}
The research work embodied in this dissertation has addressed the problem of intrusion detection in WSN with accuracy of about 99\% by using a neural network model and dataset to make neural network learn. Various aspects of the research problem are investigated and flaws in WSN security, IDS and its components, a detailed review of different IDS to tackle various attacks has been presented in earlier sections. The main findings are summarized below.
\begin{itemize}
	\item In the process of analyzing different detection techniques, it is observed that anomaly-based detection methods normally use statistical algorithms whereas signature-based detection methods generally uses knowledge-based algorithms.
	\item It can be concluded that hybrid systems are more efficient in terms of detection rate with low false positive rate as compared to the anomaly and signature-based techniques.
\end{itemize}
% The detection method proposed in this work uses a dataset and neural network which has accuracy of about 99\% on training and testing data.
\section{Scope for future study}
The proposed method works fine for Grayhole attack, Blackhole attack, Scheduling attack, Flooding attack. But, there are few limitations, as listed below, in this work which can be addressed in future.
\begin{itemize}
	\item Currently, proposed system is only able to cope with 4 types of attack. This research work can be extended by generating and collecting more data of different attack categories.
	\item Dataset creation can have more features by observing/monitoring the Sensor nodes behavior closely.
	\item Number of observation vector of attacks can be more to detect any attack more accurately by training the neural network and adding more hidden layers.
	\item The proposed approaches uses only feed forward network for classification, back-propagation algorithm can be implemented to achieve more accuracy.
	
\end{itemize}


