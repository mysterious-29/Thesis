\vspace{4.0\baselineskip}
\textbf{\hspace{4.cm}
		\LARGE \textbf{ABSTRACT}}

\vspace{15mm}
%\noindent\rule{\linewidth}{2pt}
	 	 	
Wireless Sensor Networks are most widely used in networking because of their large application domain. These networks have appealing features, like multihop wireless communication, deployment in a hostile unprotected environment, low installation cost, auto-configurable etc. Wireless sensor networks are used in almost every field because wireless sensors are very easy to deploy and maintain where wired surveillance is very difficult. Sensors are small devices deployed in the unprotected region and are vulnerable to attacks. Its large application domain and deployment in unprotected and unattended environment causing network services to be compromised. Security mechanisms, cryptography etc. can provide security to outside attacker but Intrusion detection system is must to detect network’s compromised nodes and to continue network services. To collect information from the surroundings sensor node senses information and follows multi-hop communication and the data reaches to sink. In these networks, there is no monitoring of information flow, hence security is a big concern. To provide security in Wireless Sensor Network operations, all kinds of intrusions should be detected and appropriate action must be taken against them in order to ensure that there is no harm done to the sensor network. Out of several detection techniques, this dissertation report focuses on signature-based, anomaly-based and hybrid-based techniques. Various detection models are examined based on certain parameters.
\par
This dissertation report presents an Intrusion Detection System in Wireless Sensor Networks. This research work uses a data-set of security attacks on wireless sensor network and classifies attacks with training set accuracy of 99.24\% with a categorical loss of 0.0271 and testing set accuracy of 99.32\% with a categorical loss of 0.0250 using Neural Network Classifier. This work can be categorized as signature based intrusion detection system and it can detect Black hole, Gray hole, Scheduling and flooding attacks. This report also summarizes various Intrusion Detection algorithms with their features which are used in Wireless Sensor Networks.
